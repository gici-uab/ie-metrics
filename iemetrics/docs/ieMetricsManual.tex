\documentclass[a4paper,10pt]{article}
\usepackage[]{graphicx}
\usepackage[]{times}
\usepackage{geometry}
\usepackage{framed}
\usepackage[printonlyused,nohyperlinks,nolist]{acronym} % Acronyms

\geometry{verbose,a4paper,tmargin=1.5cm,bmargin=2cm,lmargin=2cm,rmargin=2cm}
\renewcommand{\baselinestretch}{1.2}

\title{Gici ieMetrics Manual \\ \small (version 1.0)}

\author{
GICI group \vspace{0.1cm} \\
\small Department of Information and Communications Engineering \\
\small Universitat Aut{\`o}noma Barcelona \\
\small http://www.gici.uab.es  -  http://gici.uab.cat/GiciWebPage/downloads.php \\
}

\date{January 2010}

\begin{document}
\maketitle

\section{Description}

This software is an implementation of various alternative metrics for hyperspectral images.
Metrics are grouped in two families of metrics: statistical and classification-based.

The family of statistical metrics are basically extensions of alternative bi-dimensional metrics from~\cite{CLM05}.
An analysis of the performance of such metrics for progressive lossy-to-lossless hyperspectral image coding is provided in~\cite{BS09a}.
The following statistical metrics are implemented: 

\begin{description}
\item[Minimum Spectral Pearson's Correlation]
$$Pearson=\min_{x,y}\left\{\rho_{x,y}\right\}$$
\item[\acf{MSS}]~\cite{RSS+01}
$$MSS=\max_{x,y}\left\{ %
\sqrt{%
\frac{|| I_{x,y} - R_{x,y} ||_2^2}{size_z}%
+%
(1 - \rho_{x,y}^2)^2%
}%
\right\}$$
\noindent
where
$$\rho_{x,y} = \frac{\sigma(I_{x,y},R_{x,y})}{\sigma(I_{x,y})\sigma(R_{x,y})}$$
and $\sigma$ is the sample variance or covariance function.

It is used to ensure class homogeneity in an unsupervised classifier. It measures changes in spectral magnitude and direction.
\item[\acf{MSA}]
$$MSA=\max_{x,y}\left\{ \cos^{-1}\left( \frac{<I_{x,y},R_{x,y}>}{||I_{x,y}||_2 \cdot ||R_{x,y}||_2} \right) \right\}$$
The \ac{MSA} quantifies the peak angular distortion. It is brightness invariant, and is usually presented in degrees.

\item[Spectral Wang-Bovik Q]~\cite{WB02,CLM05}
$$Q_\lambda = \min_{x,y}\left\{ Q(I_{x,y},R_{x,y}) \right\}$$
$$Q_{stack} = \min_z\left\{ Q(I_{z},R_{z}) \right\}$$
$$Qm = Q_\lambda \cdot Q_{stack}$$
\noindent where
$$ Q(U,V) = \frac{4 \sigma(U,V) \mu(U) \mu(V) }{\left(\sigma(U)^2+\sigma(V)^2\right)\left(\mu(U)^2+\mu(V)^2\right)} $$

\noindent and $\mu$ is the mean function.

\item[Spectral Fidelities]~\cite{EF95,CLM05}

$$ F_{cube} = 1 - \frac{||I-R||_2^2}{||I||_2^2} $$

$$ F_\lambda = \min_{x,y} \left\{ \frac{||I_{x,y}-R_{x,y}||_2^2}{||I_{x,y}||_2^2} \right\} $$
$$ F_{stack} = \min_{z} \left\{ \frac{||I_{z}-R_{z}||_2^2}{||I_{z}||_2^2} \right\} $$

It is intended to evaluate the distortion in the three following properties: correlation, luminance, and contrast.
\end{description}

Two classification-based measures are also implemented:

\begin{description}
 \item[k-Means classification]
k-Means is a very common clustering approach~\cite{Mac67}. 
The following classification distances are used: the spectral angle, the Euler distance, and the Manhattan distance.
The spectral angle is usually selected as classification distance for its brightness invariance.
The number of desired cluster is set to 10.

 \item[\ac{RX} anomaly detection]
This is also a very common procedure in remote sensing~\cite{RY90}. While the direct application is very straightforward, it requires the
inverse of the spectral covariance matrix, which does not always exist. We use an alternative method based on the computation of the Mahalanobis distance in the \ac{KLT} space~\cite{MJM00}. As for the threshold selection, we consider the top 1\% locations to be anomalies.

\end{description}

\section{Requirements}

This software is programmed in Java, so you might need a JAVA Runtime Environment(JRE) to run this application.
We have used SUN JAVA 1.5. 

\begin{description}
\item[JAI] The Java Advanced Imaging (JAI) library is used to load and save images in formats
other than raw or pgm. The JAI library can be freely downloaded from \emph{http://java.sun.com}.
\textbf{Note:} You don't need to have this library installed in order to compile the source code.

\item[GSL] Eigendecomposition functions are from the GNU Scientific Library (GSL) and have been translated into Java.
The authors of the of original code are Gerard Jungman and Brian Gough. (see source files for details)
\end{description}

\section{Usage}

The application is provided in a single file, a jar file (\emph{dist/iemetrics.jar}), that contains the application.
Along with the application, the source code is also provided. If you need to rebuild the jar file, you can use the \texttt{ant} command.

To launch the application you can use the following command: 

\begin{framed}
\texttt{\$ java -Xmx1200m -jar dist/iemetrics.jar --help}
\end{framed}

In a GNU/Linux environment you can also use the shell script \texttt{iemetrics} situated at the root of the iemetrics directory. 

\begin{framed}
\texttt{\$ ./iemetrics --help}
\end{framed}

The output is a double-colon-delimited list with the following fields:
\begin{framed}%
\vspace{-1em}%
\begin{itemize}
\item Maximum Spectral Similarity
\item Maximum Spectral Angle
\item Maximum Spectral Information Divergence
\item Minimum Pearsons Correlation
\item Wang Index Lambda
\item Wang Index Stack
\item Wang Index Both
\item Eskicioglu Cube Fidelity
\item Eskicioglu Spectral Fidelity
\item Eskicioglu Stack Fidelity
\item POC k-MEANs SAM
\item POC k-MEANs Dot
\item POC k-MEANs Euler
\item POC k-MEANs Manhattan
\item POC ISODATA SAM (disabled)
\item POC ISODATA Dot (disabled)
\item POC ISODATA Euler (disabled)
\item POC ISODATA Manhattan (disabled)
\item POC RX
\end{itemize}%
\vspace{-1em}%
\end{framed}

Two examples of usage are provided below:

\begin{itemize}
\item Compare two images using the alternative metrics.
\begin{framed}%
\vspace{-1em}%
\begin{verbatim}
$ ./iemetrics -i1 "$INFILE-16bpppb-bigendian.raw" -ig1 $Z $Y $X 3 0 \
              -i2 "$OUTFILE-16bpppb-bigendian.raw" -ig2 $Z $Y $X 3 0
\end{verbatim}%
\vspace{-1em}%
\end{framed}

\item Compare two images using the alternative metrics, and dump some visual results of the comparison.
\begin{framed}%
\vspace{-1em}%
\begin{verbatim}
$ iemetrics -i1 "$INFILE-16bpppb-bigendian.raw" -ig1 $Z $Y $X 3 0 \
            -i2 "$OUTFILE-16bpppb-bigendian.raw" -ig2 $Z $Y $X 3 0 \
            -dr "$PARTIAL_RESULT_DUMP_FOLDER/"
\end{verbatim}%
\vspace{-1em}%
\end{framed}
\end{itemize}

\section{Notes}

If you need further assistance, you might want to contact us directly.

\bibliographystyle{IEEEtran}
\bibliography{IEEEabrv,biblio}

% Acronyms
\begin{acronym}

\acro{R-D}{Rate-Distortion}

\acro{ERM}{Elementary Reversible Matrix}
\acro{TERM}{Triangular Elementary Reversible Matrix}
\acro{SERM}{Single-row Elementary Reversible Matrix}

\acro{FFT}{Fast Fourier Transform}
\acro{DCT}{Discrete Cosine Transform}
\acro{KLT}{Karhunen-Lo{\^e}ve Transform}
\acro{AKLT}{Approximate Karhunen-Lo{\^e}ve Transform}
\acro{DPCM}{Differential Pulse Code Modulation}
\acro{PLT}{Prediction-based Lower triangular Transform}
\acro{ICA}{Independent Component Analysis}
\acro{PCA}{Principal Component Analysis}
\acro{DWT}{Discrete Wavelet Transform}
\acro{IWT}{Integer Wavelet Transform}
\acro{RKLT}{Reversible Karhunen-Lo{\^e}ve Transform}
\acro{VolPCRD}{Volumetric Post-Compression Rate-Distortion}

\acro{MSE}{Mean Squared Error}
\acro{PSNR}{Peak Signal-to-Noise Ratio}
\acro{SNR}{Signal-to-Noise Ratio}
\acro{MAE}{Mean Absolute Error}

\acro{MINLAB}{Minimum Noise Ladder-Based Structure}

\acro{BIFR}{Band-Independent Fixed Rate}
\acro{3d-DWT}{3-Dimensional Discrete Wavelet Transform}
\acro{VolPCRD}{Volumetric Post-Compression Rate-Distortion}
\acro{PCRD}{Post-Compression Rate-Distortion}

\acro{MCT}{Multi-component Transform}
\acro{ML}{Multi-level}

\acro{CASI}{Compact Airborne Spectrographic Imager}
\acro{CREAF}{Center for Ecological Research and Forestry Applications}
\acro{NASA}{National Aeronautics and Space Administration}
\acro{AVIRIS}{Airborne Visible/Infrared Imaging Spectrometer}
\acro{CT}{Computed Tomography}

\acro{JPEG2000}{}
\acro{TCE}{Tarp-based coding with Classification for Embedding}
\acro{3d-TCE}{3-Dimensional Tarp-based coding with Classification for Embedding}

\acro{SVD}{Singular Value Decomposition}
\acro{IEEE}{Institute of Electrical and Electronics Engineers}
%\acro{RD}{Rate-Distortion}
\acro{CR}{Compression-Ratio}
\acro{ML}{Multi-Level}
\acro{GIS}{Geographic Information Systems}
\acro{MRI}{Magnetic Resonance Imaging}

\acro{POC}{Preservation of Classification}
\acro{MSS}{Maximum Spectral Similarity}
\acro{MSA}{Maximum Spectral Angle}

\acro{PLL}{Progressive Lossy-to-Lossless}
\acro{RS}{Remote Sensing}
\acro{IEM}{Information Extraction Measure}

\acro{RX}{Reed Xiaoli}

% Fakes for plurals to be used with facp. DO NOT USE OTHERWISE OR THEY WILL BE LISTED!
\acro{ERMs}{Elementary Reversible Matrices}
\acro{TERMs}{Triangular Elementary Reversible Matrices}
\acro{SERMs}{Single-row Elementary Reversible Matrices}
\acro{MCTs}{Multi-component Transforms}

\end{acronym}

\end{document}
